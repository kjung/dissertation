\chapter{Prediction of Delayed Wound Healing in Outpatient Wound Care
  Centers}

\section{INTRODUCTION}

Chronic wounds are those that fail to heal in a timely manner (1), and
affect an estimated 6.5 million people in the United States (2\% of the
population) (2).  Chronic wounds are at increased risk of
complications, such as amputation and infection, which can have a
severe negative impact on patient well-being.  The cost of treating
these wounds is high – up to $50 billion annually (3-5) – and the
incidence of chronic wounds is expected to increase due to an aging
population and rising risk factors such as diabetes and obesity (6).
Knowing in advance that a given wound is likely to be problematic
despite standard care could enable care providers to make better
decisions about treatment options such as early intervention with
hyperbaric oxygen therapy (HBOT) or negative pressure wound therapy
(NPWT) that may improve outcomes (7-13).

Previous work has attempted to identify factors of prognostic value in
predicting delayed wound healing, but most of these studies were
restricted to specific wound types such as venous leg ulcers (14-19).
They used patient cohorts of modest size and drawn from single sites
or enrolled in clinical trials, limiting the generalizability of these
results to the diversity of patients and wound types seen in clinical
practice (20).  Furthermore, these models have not yet been validated
in independent datasets with respect to either discriminatory power or
calibration.

Among the best work to date is that of Margolis et al. (14, 15) who
developed prognostic models for venous leg ulcers and diabetic
neuropathic foot ulcers using data from tens of thousands of patients
across geographically diverse outpatient wound care centers, and
carefully validated the models, achieving excellent calibration but
only modest discriminative power (AUCs ranging from 0.63 to 0.71).
These results led to the study by Kurd et al. (21), a clustered
multicenter trial demonstrating that providing prognostic information
from these models to clinicians improved healing rates even without
specific guidance about treatment options.

In this work, we report the development and validation of a novel
prognostic model that uses data routinely collected over the first two
weeks of care at outpatient wound centers.  Our model achieves an AUC
of 0.863 (95\% confidence interval 0.857-0.869) on held out data and
provides well-calibrated probabilities for delayed wound healing.  Our
model is constructed and validated on a dataset comprising tens of
thousands of patients with over a hundred thousand wounds from
geographically diverse wound care centers.  All wound types are
considered in this work, and thus the model is applicable to the full
diversity of wounds seen in clinical practice.


\section{METHODS}

\subsection{Dataset}

Figure 1 provides an overview of the dataset, and Supplemental
Materials S1 contains a full list of the variables collected, along
with basic patient and wound statistics.  In brief, the dataset was
compiled from data collected at 68 Healogics wound care centers in 26
states from 2009 through 2013.  This includes data on 180,696 wounds
from 59,953 patients, with wounds receiving standard care, and
assessed weekly until healed or resolved, or treatment was
discontinued.  Patient information includes demographic information
along with ICD9 codes at each encounter, zip code, and insurance
provider.  Wound information included quantitative wound
characteristics such as length and width, along with categorical wound
characteristics such as “rubor” and “erythema”.  There were 40 wound
types (e.g., Pressure Ulcer, Venous Ulcer) and 37 anatomic locations.
The mean duration of care was 52.2 days.  Overall, 11.6\% of wounds
exhibited delayed wound healing, defined as 15 or more weeks to
closure.

\subsection{Model development}

Model development is summarized in Figure 2.  We used data from the
first and second wound assessments (weeks one and two) to construct
predictors of delayed wound healing.  We chose delayed wound healing
as the outcome because long term, non-healed wounds are a significant
source of morbidity for patients, increase cost to the healthcare
system and, in a capitated payment model, a significant risk for the
care provider. These data were augmented with additional predictors
that captured wound burden at the first assessment, such as number of
patient wounds and total wound surface area, along with predictors
that measure the change in quantitative metrics between the first and
second assessments.  Finally, we linked patient zip codes to
socio-economic variables from the American Community Survey (23).
After removal of data that containing outlier values, we were left
with 149,829 wounds and 53,198 patients.

The dataset was randomly split into 2 parts – training (N = 119,869)
and test (N = 29,960).  Quantitative predictors were standardized
using mean and variance estimates from the training set; these
parameters were also used to standardize the test data.  Categorical
variables were encoded as binary values.  In all, 1042 predictors were
used, 911 of which were binary (0 or 1) encodings of categorical
variables.

We fit several models on the training data and evaluated their
performance on a held out subset of the training data.  We found that
the best single model was a generalized boosted tree model (24, 25).
This model was then trained on the full training dataset and evaluated
on the test data.  The ability of the model to predict delayed wound
healing was measured by area under the receiver-operator curve (AUC),
with 95\% confidence intervals estimated by bootstrapping (26).  Model
calibration (i.e., the agreement of the predicted probabilities of
delayed wound healing to observed frequencies) was measured using
Brier reliability (27).  Further details may be found in Supplemental
Materials S2.

\section{RESULTS}

A prognostic model for delayed wound healing should discriminate
between normal and delayed healing wounds, and provide accurate
probabilities for those outcomes (28, 29).  We evaluated
discrimination by AUC, and calibration by Brier reliability (Figure
3).  The AUC summarizes the ability of the model to discriminate
between normal and non-healing wounds over the range of tradeoffs
between sensitivity and specificity.  An AUC of 0.5 means the model
performs no better than random guessing, while an AUC of 1 means the
model discriminates perfectly.

Our model achieved an AUC of 0.863 (95\% confidence interval
0.857-0.869) over all wound types.  For the two wound types previously
examined by Margolis et al. (14, 15) (venous leg ulcers and diabetic
neuropathic ulcers), we achieve AUCs of 0.871 and 0.835 respectively,
a significant improvement over the previously reported best results of
0.71 and 0.70.  Performance in other wound types well represented in
the test set (N > 500) ranged from 0.811 for lower extremity diabetic
wounds to 0.934 for partial thickness burns.

Brier reliability measures agreement between predicted probabilities
and observed frequencies for stratified samples on a scale between 0
and 1, with small values indicating good agreement and 0 indicating
perfect agreement.  Test cases were stratified based on their
predicted probability of delayed wound healing.  The model scored a
Brier reliability of 0.00018 indicating that the predicted
probabilities in each strata closely matched observed frequencies of
delayed wound healing.

\section{DISCUSSION}

We developed a prognostic model for delayed wound healing outcomes
that uses data routinely captured in EHRs during the first two weeks
of care at wound centers.  The model achieves best-to-date
discriminative power and calibration on validation data.  Unlike
previous work, our model is also applicable to the full range of wound
types.  Furthermore, by changing the cutoff threshold, it is possible
to make trade offs between positive predictive value and sensitivity
(recall) for specific uses of the model (Figure 3b).  For instance, it
may be desirable to trade off low positive predictive value for a high
sensitivity when making decisions regarding referral to specialized
wound care centers.  Once referred to a wound center, however, it may
be desirable to have a high positive predictive value and sacrifice
sensitivity when making decisions about high cost interventions that
may improve outcomes in potentially problematic cases.  Because the
model outputs are well-calibrated probabilities, they can be used
directly to make informed referral or intervention decisions that
helps care providers and patients.

Our approach does have limitations. Although our model was developed
on data from 68 geographically distributed wound care centers in 26
states, it may not generalize to centers other than those that
participated in this study.  Indeed, we observe considerable
inter-center variability in practices and patient populations.
Second, it is possible that the model may not generalize to cases
outside wound care centers, limiting its utility for making referral
decisions.  Third, in contrast with previous work, we have traded
simplicity for accuracy.  The simple models by Margolis et al. can be
applied by basic arithmetic.  In exchange, however, we have achieved
significantly higher accuracy.  We would argue that in the age of
widespread and growing adoption of EHRs, it is time to shift some of
the burden of such prognostication to automated systems.

These limitations notwithstanding, our model, to the best of our
knowledge, is the first that has been developed and validated on the
full diversity of patients and wound types seen in clinical practice.
It achieves significantly better discrimination than those of previous
efforts, and is a case study for the meaningful secondary use of
routinely collected EHR data to improve treatment strategies.

